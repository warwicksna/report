\subsection{Centrality}
Following from identifying communities within a social network, it is also apparent that establishing the most \emph{central} or \emph{important} person is within a community. This person is most likely going to have a large influence over the members of the community.

\subsubsection{Degree Centrality}
The degree centrality is a simple measure to produce for a social network. For an undirected graph (relationships are symmetric), the degree centrality for a vertex, \emph{v}, is simply the number of edges connected to it. For a directed graph, say the network of Twitter\footnote{\url{http://twitter.com/}} users following, and being followed by others, two measures are defined, the in-degree and out-degree. These are simply the number of incoming and outgoing edges respectively.

Whilst a fairly simple measure to compute, degree centrality can be useful. A person with a high degree centrality is likely to hold a larger influence amongst the people they share a connection with, as they are more popular. \cite[p.~169]{newman10} suggests that the number of citations an academic paper receives can be used as its in-degree centrality, and is a crude measure of whether the paper was influential and as such had an impact on scientific research.

\subsubsection{Eigenvector Centrality}
The eigenvector centrality measure identifies important vertices within a graph

\subsubsection{PageRank}
The PageRank algorithm \cite{pagerank} developed by Google is a further extension of the eigenvector centrality measure, and has similarities to the Katz centrality 

\subsubsection{Closeness Centrality}
Closeness centrality is a measure of the average distance from a vertex to other vertices. A geodesic path between two vertices is the minimum number of edges required to connect them within a graph. The closeness centrality defined in equation \ref{eq:closenesscentrality1} is the sum of the length of all geodesic paths, $d_{ij}$, from a vertex, $i$, to all other vertices, $j$:

\begin{equation}
l_i = \frac{1}{n}\sum_{j} d_{ij}
\label{eq:closenesscentrality1}
\end{equation}

A slight alternative is to exclude the geodesic path $d_{ij}$ when $i = j$ because this is trivially 0, and a vertex's influence on itself is not relevant to the calculation of centrality:

\begin{equation}
l_i = \frac{1}{n-1}\sum_{j(\neq i)} d_{ij}
\label{eq:closenesscentrality2}
\end{equation}

Closeness centrality defined in equations \ref{eq:closenesscentrality1} and \ref{eq:closenesscentrality2} produce results differently to other measures of centrality, whereby a lower value for $l_i$ indicates that vertex $i$ is more central within the network. Because of this, closeness centrality can also be defined as:

\begin{equation}
C_i = \frac{1}{l_i} = \frac{n}{\sum_{j(\neq i)} d_{ij}}
\label{eq:closenesscentrality3}
\end{equation}

Equation \ref{eq:closenesscentrality3} produces results more consistent with other measures of centrality as a higher value of $C_i$ indicates vertex $i$ is more central. This equation is simply the inverse of equation \ref{eq:closenesscentrality1}.

Within a social network, a person with a high closeness centrality, defined by equation \ref{eq:closenesscentrality3}, could find that their opinions reach others in the community faster than someone with a lower closeness centrality \cite{newman10}.

\subsubsection{Betweeness Centrality}

