\section{Emergent Behaviour}

The simulation will be written in Java.
Java was chosen for many reasons, such as:
\begin{itemize}
    \item it is available on all of the most popular computing platforms;
    \item it produces cross-platform applications;
    \item it is easy to package dependencies in one bundle;
    \item there is a wealth of third-party libraries and tools which can be
used.
    \item it has good performance, which is necessary given the scale of the
simulations.
\end{itemize}

The simulation will use the Java Universal Network/Graph Framework
(JUNG),
as the base for the graph implementation.
JUNG is an open-source graph library and contains a lot of data
structures for representing graphs and routines for manipulating the
graphs.
JUNG was chosen for many reasons, including:
\begin{itemize}
    \item it is able to generate many different types of graphs;
    \item it is mature, and has no bugs which would impair the results;
    \item writing bespoke code to handle graphs would be challenging to get
right;
    \item it includes many useful algorithms that can be useful in analysing the
graphs.
\end{itemize}

A `wrapper' class will be written around the JUNG {\tt graph} class in
order to introduce additional functionality that will be required in the
simulation.

Graphs will be generated using JUNG's random graph generators.

Every vertex in the graph will represent an agent in the simulation.
To allow individual vertices to be randomly access in a fast manner,
each vertex will have a unique identifier and be stored in a lookup
data-structure.

Each agent will have a tag and tolerance instance variable to be able to
represent an agent in the graph.
Every agent will also have a {\tt score} instance variable which will be
used to keep track of the success of the agent.
When the agent donates, the cost of donation, $c$, will be subtracted
from the agent's score variable.
When the agent receives a donation, the benefit, $b$, will be added to
the agent's score variable.
This approach allows the success of a given agent to be compared to the
success of another agent by comparing the score variables of each agent.

A class will be created to control the parameters of a simulation---such
as the number of generations to simulate, or the number of interaction
pairings per agent per generation---and to allow multiple simulations to
be run with different parameters in order to compare the results.

Parameters which will be able to be set are:

\begin{tabular}{llll}
    Name & Type & Default Value & Remarks \\
    \hline
    Generations & Integer & 1,000 & Must be greater than 0\\
    Cost & Double & 0.1 & Must be greater than 0 \\
    Benefit & Double & 1 & Must be greater than Cost \\
    Context Influence & Double & 0.6 & Must be in $[0, 1]$ \\
    Learn Probability & Double & 0.1 & Must be in $[0, 1]$ \\
    Rewire Proportion & Double & 0.5 & Must be in $[0, 1]$ \\
    Cheater Proportion & Double & 0.1 & Must be in $[0, 1]$ \\
    Mutation Probability & Double & 0.01 & Must be in $[0, 1]$ \\
    Graph Type & Graph Type & Random & Null values not permitted \\
    Rewire Strategy & Rewire Strategy & Null & Null values are permitted
    \\
\end{tabular}

The simulation will be started by passing in a set of parameters.
The simulation will then generate a graph and set the initial state of
the simulation according to the parameters passed in.
The simulation will then be run.

\begin{algorithmic}
\While {iterations remain}
    \State Donation()
    \State Reproduction()
\EndWhile
\end{algorithmic}

% \includegraphics{./img/mp/AgentVertex.1}
