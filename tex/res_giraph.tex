\subsection{Giraph}
Giraph is a free implementation of Google's Pregel framework.

\subsubsection{Pregel}
Pregel is a framework developed at Google for large-scale graph processing \cite{pregel}

\subsubsection{Giraph}
Giraph is a graph processing framework, inspired by Pregel, initially developed at Yahoo!. It has since become an Apache Incubator project\footnote{\url{http://incubator.apache.org/giraph/}}.

Giraph is a framework which was designed for large-scale graph processing on Hadoop. There are a number of existing solutions to achieve large-scale graph processing, however these also have their own problems.

Graph algorithms can be executed as a sequence of map-reduce jobs in Hadoop, but this suffers from the overheads of repeatedly launching these jobs and the map-reduce model is not a good fit for graph algorithms. Pregel itself has problems in that it requires a separate computing infrastructure, and is also unavailable to non-Google employees. The Message Passage Interface can also be used for graph processing, however it lacks any form of fault tolerance and is consider too generic \cite{giraphtalk}.

\lstset{language=Java,caption={PageRank algorithm in Giraph},label=giraphpagerank,tabsize=2,breaklines=true,breakatwhitespace=true,frame=single}
\begin{lstlisting}
public class SimplePageRankVertex extends HadoopVertex<LongWritable, DoubleWritable, FloatWritable, DoubleWritable> {

	public void compute(Iterator<DoubleWritable> msgIterator) {
		double sum = 0;
		
		while (msgIterator.hasNext()) {
			sum += msgIterator.next().get();
		}
		
		setVertexValue(new DoubleWritable((0.15f/getNumVertices()) + 0.85f * sum);
		
		if (getSuperstep() < 30) {
			long edges = getOutEdgeIterator().size();
			
			sendMsgToAllEdges(new DoubleWritable(getVertexValue().get() / edges));
		} else {
			voteToHalt();
		}
	}
}					
\end{lstlisting}