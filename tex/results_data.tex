\section{Data Sets}
\label{sec:results_datasets}
This section details results we observe about various datasets. We have performed analysis on the Twitter datasets which were collected on the English Defence League and Unite Against Fascism.

The Stanford Network Analysis Project website, which hosts various datasets we are performing analysis on, provides statistics about each dataset available, including the number of triangles within the dataset (assuming the network is undirected), and the average clustering coefficient. The average clustering coefficient is the network local clustering coefficient.

\subsection{English Defence League}
simple enough to do

\subsection{Unite Against Fascism}
simple enough to do

\subsection{Enron}
One dataset which we needed to check the performance of DSNAT with was the Enron email network\footnote{\url{http://snap.stanford.edu/data/email-Enron.html}}. This is a dataset containing nodes representing email addresses, and connections existing between the nodes if at least one email was sent between them. The dataset contains 36,692 nodes and 367,662 edges.

For the Enron dataset, SNAP lists the number of triangles existing within the network as 727,044 and the average clustering coefficient as 0.4970. 

\subsubsection{3-Clique}
As stated previously, the identification of 3-cliques within a graph is also known as triangle identification. Using the triangle identification algorithm, we were able to mutate the network such that the only nodes and edges remaining within the graph were those nodes constructing at least one triangle two other nodes.

The number of triangles existing within the network is simply the sum of the values stored at all of the nodes within the network, as the node with the highest ID in the triangle contains the count of the triangle.

After summing the number of triangles, we also arrive at the same figure of 727,044 triangles present within the Enron network. Table \ref{tab:enrontriangles} shows the node ID of email addresses present in the Enron network forming the most triangles.

\begin{table}%
\centering
\begin{tabular}{|c|c|}
\hline
Node ID & Number of Triangles \\
\hline
136 & 17744 \\
195 & 15642 \\
76 & 13767 \\
370 & 13671 \\
273 & 13401 \\
1028 & 13064 \\
416 & 11957 \\
292 & 11415 \\
140 & 11265 \\
175 & 10775 \\
734 & 10621 \\
458 & 9217 \\
1139 & 8780 \\
520 & 8662 \\
188 & 8450 \\
444 & 8199 \\
478 & 8068 \\
241 & 7874 \\
639 & 7783 \\
823 & 7590 \\
\hline
\end{tabular}
\caption{Top 20 nodes forming triangles}
\label{tab:enrontriangles}
\end{table}

\subsubsection{Local Clustering Coefficient}
There are 12499 nodes with a local clustering coefficient of 1, which indicates that the direct neighbours of these nodes are also themselves direct neighbours of each other.

There are also 12240 nodes which have a local clustering coefficient of 0, which indicates that nearly one third of the nodes within the network do not form any triangles, and as such are likely to be email addresses external to Enron.

\subsubsection{Network Local Clustering Coefficient}
The network local clustering coefficient is the average of all local clustering coefficients. For the Enron network, DSNAT returns the result of performing the network local clustering coefficient algorithm as: {\tt 0.49698255959950327}.

This value is near identical to the average clustering coefficient provided by with the dataset by SNAP, so shows that our tool, and these implemented algorithms are working correctly.

\subsubsection{PageRank}
Table \ref{tab:enronpagerank} shows the top twenty nodes in the Enron email network, as ranked by the PageRank algorithm.

What is interesting from the results of running PageRank is that eleven of the top twenty nodes ranked by PageRank also feature within the top twenty nodes connected to the highest number of triangles. These nodes are: 76, 136, 140, 195, 273, 292, 370, 458, 823, 1028 and 1139

\begin{table}%
\centering
\begin{tabular}{|c|c|}
\hline
Node ID & PageRank \\
\hline
5038 & 0.013662506409208037 \\
273 & 0.0032385323056388035 \\
140 & 0.0029987944502075873 \\
458 & 0.0029642208948891463 \\
588 & 0.0029383621879706633 \\
566 & 0.0029094845599689167 \\
1028 & 0.0027870875267837626 \\
1139 & 0.0025456338556696773 \\
370 & 0.0023498594485031215 \\
893 & 0.002199691021020677 \\
195 & 0.002105444989925047 \\
823 & 0.002006310315755743 \\
136 & 0.001883568213027711 \\
286 & 0.0016956990694867655 \\
95 & 0.001694141051220431 \\
5069 & 0.0016638405778172485 \\
5022 & 0.0016023313239255674 \\
76 & 0.0015252624602667306 \\
292 & 0.0015224551091379053 \\
1768 & 0.0015039521949740957 \\
\hline
\end{tabular}
\caption{Top 20 nodes ranked by PageRank values}
\label{tab:enronpagerank}
\end{table}

\subsection{Amazon Product Co-Purchasing}
The Amazon product co-purchasing network\footnote{\url{http://snap.stanford.edu/data/amazon0302.html}} is data collected from the crawling of the Amazon website. It is based on the \emph{Customers Who Bought This Iterm Also Bought} feature of the Amazon website. A node within the network represents a product on sale, and a directed edge between nodes represents that the product has been co-purchased with the product the edge connects to. The data was collected in March 2003 \cite{snap}.

The datasets contains 262111 nodes and 1234877 edges, and is listed with an average clustering coefficient of 0.4240 and 717719 triangles present.

\subsection{Wikipedia Talk}
\url{http://snap.stanford.edu/data/wiki-Talk.html}

\subsection{Patent citation}
\url{http://snap.stanford.edu/data/cit-Patents.html}

\subsection{LiveJournal}
needs the cluster working. I think possibly too big

\subsection{Twitter}
needs the cluster working. This is probably too big for the cluster.




% 7 workers, different graph sizes

% enron - 36,692	- 367,662
% 58.955
% 58.401
% 57.209

% amazon - 262,111 - 1,234,877
% 137.001
% 134.949
% 138.012

% wikitalk - 2,394,385 - 5,021,410
% 585.326
% 636.433

% patents - 3,774,768 - 16,518,948