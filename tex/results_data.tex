\section{Data Sets}
\label{sec:results_datasets}
This section details results we observe about various datasets. We have performed analysis on the Twitter datasets which were collected on the English Defence League and Unite Against Fascism.

The Stanford Network Analysis Project website, which hosts various datasets we are performing analysis on, provides statistics about each dataset available, including the number of triangles within the dataset (assuming the network is undirected), and the average clustering coefficient. The average clustering coefficient is the network local clustering coefficient.

\subsection{English Defence League}
With the data collected on the English Defence League, we applied some of the algorithms implemented to perform analysis.

\subsubsection{PageRank}
There is a bias present within the results of the performing the PageRank algorithm on the data collected relating to the EDL. As can be seen in both tables \ref{tab:edlpagerank} and \ref{tab:edltweetpagerank}, the Twitter account with the highest PageRank is the edlsupportgroup. This is unsurprising given that the edlsupportgroup account was chosen as the root node for the data collecting.

The most surprising outcome from performing PageRank is that TwitPic\footnote{\url{http://twitpic.com/}} and TwitPic's founder Noah Everett are ranked second and third. TwitPic is a service to share photos and videos using Twitter, so it is likely that many of the accounts of which data was collected make use of TwitPic resulting in it's high placement.

The majority of the accounts listed in tables \ref{tab:edlpagerank} and \ref{tab:edltweetpagerank} are however related to the EDL, with an approximate equal number for and against the EDL.

\begin{table}[htbp]%
\centering
\begin{tabular}{|l|l|}
\hline
Twitter Name & PageRank \\
\hline
edlsupportgroup & 0.08416403924447517 \\
TwitPic & 0.05975756896253487 \\
noaheverett & 0.058158047141795156 \\
everythingEDL & 0.0257624790515872 \\
Official\_EDL & 0.019653741734420287 \\
hopenothate & 0.016939681748134754 \\
TommyRobinson3 & 0.016757784171303376 \\
leicspolice & 0.015009582286344358 \\
jihadwatchRS & 0.011358838738844794 \\
OneMillionUtd & 0.010758557526296803 \\
british\_freedom & 0.010247425883370982 \\
USoBritain & 0.008615666297945848 \\
EDLGommy & 0.008447834177284598 \\
edlnews & 0.0082414322437749 \\
SuptPayneWMP & 0.007883869657789162 \\
KateUSoB & 0.007613706376677299 \\
Englandstruth & 0.007590315713051853 \\
PboroCops & 0.007534644943547635 \\
exposetweets & 0.007522329904304726 \\
MuslimCouncil & 0.007313457877227599 \\
\hline
\end{tabular}
\caption{PageRank on EDL data}
\label{tab:edlpagerank}
\end{table}

\begin{table}[htbp]%
\centering
\begin{tabular}{|l|l|}
\hline
Twitter Name & PageRank \\
\hline
edlsupportgroup & 0.09027442551864633 \\
TwitPic & 0.05159443925208596 \\
noaheverett & 0.050126221757179935 \\
everythingEDL & 0.02798669165838786 \\
Official\_EDL & 0.025266481999928907 \\
leicspolice & 0.020755126860433737 \\
TommyRobinson3 & 0.019016391652185494 \\
hopenothate & 0.014987830036646874 \\
edlnews & 0.012970290691972114 \\
jihadwatchRS & 0.012497551750545299 \\
brianoflondon & 0.011623243499760285 \\
MuslimCouncil & 0.011357910646713614 \\
british\_freedom & 0.011315839428723318 \\
Englandstruth & 0.011155872770274453 \\
SuptPayneWMP & 0.011037133218573704 \\
EDLGommy & 0.009545490480397533 \\
PboroCops & 0.009501841973798724 \\
ChrisWebbo & 0.009457741742444896 \\
snidey\_UK & 0.008122928964687613 \\
OneMillionUtd & 0.007613901445608443 \\
\hline
\end{tabular}
\caption{PageRank on EDL considering edges as tweets}
\label{tab:edltweetpagerank}
\end{table}

\subsubsection{Vertex Degrees}
As the EDL dataset is directed, we analyse both the in-degree and out-degree of each vertex and present the results in Tables \ref{tab:edlin} and \ref{tab:edlout}

\begin{table}[htbp]%
\centering
\begin{tabular}{|l|l|}
\hline
Twitter Name & In-Degree \\
\hline
220331507 & 1559 \\
410224577 & 496 \\
374712154 & 468 \\
201247027 & 426 \\
15610459 & 209 \\
332320606 & 205 \\
265287278 & 194 \\
256943061 & 161 \\
142082006 & 138 \\
353657708 & 131 \\
12925072 & 130 \\
234309657 & 124 \\
199808054 & 91 \\
245761263 & 84 \\
440109993 & 79 \\
46760864 & 79 \\
19985444 & 78 \\
13058232 & 76 \\
279714026 & 76 \\
227954333 & 75 \\
\hline
\end{tabular}
\caption{In-Degrees for EDL dataset}
\label{tab:edlin}
\end{table}

\begin{table}[htbp]%
\centering
\begin{tabular}{|l|l|}
\hline
Twitter Name & Out-Degree \\
\hline
353657708 & 204 \\
199808054 & 133 \\
495874697 & 128 \\
264238936 & 93 \\
142082006 & 92 \\
373843347 & 87 \\
245905821 & 83 \\
47223385 & 80 \\
392058404 & 77 \\
185624680 & 75 \\
94855635 & 72 \\
21142055 & 68 \\
279714026 & 67 \\
409492198 & 65 \\
55935057 & 63 \\
415038513 & 49 \\
448017421 & 49 \\
19485684 & 48 \\
509439451 & 47 \\
479074142 & 44 \\
\hline
\end{tabular}
\caption{Out-Degrees for EDL dataset}
\label{tab:edlout}
\end{table}

\subsubsection{Clustering Coefficients}
Due to the approach of collecting the data, we felt that the results of the clustering coefficients would be quite heavily biased. 

As expected the results of the network local clustering coefficient algorithm is quite high, with a value of \verb/0.68038719629578/ which indicates that the nodes within the network cluster together. This is unsurprising given the method used to collect data.

\subsection{Unite Against Fascism}
The analysis performed on this data is similar to the analysis performed on the data collected about EDL.

\subsubsection{Clustering Coefficients}
As the data collected about the UAF was collected in an identical manner to the EDL, we again felt that the results of the clustering coefficients would again be quite biased.

\subsection{Enron}
One dataset which we needed to check the performance of DSNAT with was the Enron email network\footnote{\url{http://snap.stanford.edu/data/email-Enron.html}}. This is a dataset containing nodes representing email addresses, and connections existing between the nodes if at least one email was sent between them. The dataset contains 36,692 nodes and 367,662 edges.

For the Enron dataset, SNAP lists the number of triangles existing within the network as 727,044 and the average clustering coefficient as 0.4970. 

\subsubsection{3-Clique}
As stated previously, the identification of 3-cliques within a graph is also known as triangle identification. Using the triangle identification algorithm, we were able to mutate the network such that the only nodes and edges remaining within the graph were those nodes constructing at least one triangle two other nodes.

The number of triangles existing within the network is simply the sum of the values stored at all of the nodes within the network, as the node with the highest ID in the triangle contains the count of the triangle.

After summing the number of triangles, we also arrive at the same figure of 727,044 triangles present within the Enron network. Table \ref{tab:enrontriangles} shows the node ID of email addresses present in the Enron network forming the most triangles.

\begin{table}%
\centering
\begin{tabular}{|c|c|}
\hline
Node ID & Number of Triangles \\
\hline
136 & 17744 \\
195 & 15642 \\
76 & 13767 \\
370 & 13671 \\
273 & 13401 \\
1028 & 13064 \\
416 & 11957 \\
292 & 11415 \\
140 & 11265 \\
175 & 10775 \\
734 & 10621 \\
458 & 9217 \\
1139 & 8780 \\
520 & 8662 \\
188 & 8450 \\
444 & 8199 \\
478 & 8068 \\
241 & 7874 \\
639 & 7783 \\
823 & 7590 \\
\hline
\end{tabular}
\caption{Top 20 nodes forming triangles}
\label{tab:enrontriangles}
\end{table}

\subsubsection{Clustering Coefficients}
There are 12499 nodes with a local clustering coefficient of 1, which indicates that the direct neighbours of these nodes are also themselves direct neighbours of each other.

There are also 12240 nodes which have a local clustering coefficient of 0, which indicates that nearly one third of the nodes within the network do not form any triangles, and as such are likely to be email addresses external to Enron.

The network local clustering coefficient is the average of all local clustering coefficients. For the Enron network, DSNAT returns the result of performing the network local clustering coefficient algorithm as: {\tt 0.49698255959950327}.

This value is near identical to the average clustering coefficient provided by with the dataset by SNAP, so shows that our tool, and these implemented algorithms are working correctly.

\subsubsection{PageRank}
Table \ref{tab:enronpagerank} shows the top twenty nodes in the Enron email network, as ranked by the PageRank algorithm.

What is interesting from the results of running PageRank is that eleven of the top twenty nodes ranked by PageRank also feature within the top twenty nodes connected to the highest number of triangles. These nodes are: 76, 136, 140, 195, 273, 292, 370, 458, 823, 1028 and 1139

\begin{table}%
\centering
\begin{tabular}{|c|c|}
\hline
Node ID & PageRank \\
\hline
5038 & 0.013662506409208037 \\
273 & 0.0032385323056388035 \\
140 & 0.0029987944502075873 \\
458 & 0.0029642208948891463 \\
588 & 0.0029383621879706633 \\
566 & 0.0029094845599689167 \\
1028 & 0.0027870875267837626 \\
1139 & 0.0025456338556696773 \\
370 & 0.0023498594485031215 \\
893 & 0.002199691021020677 \\
195 & 0.002105444989925047 \\
823 & 0.002006310315755743 \\
136 & 0.001883568213027711 \\
286 & 0.0016956990694867655 \\
95 & 0.001694141051220431 \\
5069 & 0.0016638405778172485 \\
5022 & 0.0016023313239255674 \\
76 & 0.0015252624602667306 \\
292 & 0.0015224551091379053 \\
1768 & 0.0015039521949740957 \\
\hline
\end{tabular}
\caption{Top 20 nodes ranked by PageRank values}
\label{tab:enronpagerank}
\end{table}

\subsubsection{Vertex Degrees}
The Enron dataset is an undirected network, and as such the in-degrees will be the same as the out-degrees. Table \ref{tab:enrondegree} contains the top twenty results for nodes with the highest number of connections.

\begin{table}[htbp]%
\centering
\begin{tabular}{|l|l|}
\hline
Node ID & Degree \\
\hline
5038 & 1383 \\
273 & 1367 \\
458 & 1261 \\
140 & 1245 \\
1028 & 1244 \\
195 & 1143 \\
370 & 1099 \\
1139 & 1068 \\
136 & 1026 \\
566 & 924 \\
823 & 908 \\
292 & 834 \\
588 & 829 \\
76 & 815 \\
416 & 791 \\
286 & 711 \\
353 & 705 \\
734 & 686 \\
851 & 666 \\
1824 & 609 \\
\hline
\end{tabular}
\caption{In-Degrees for EDL dataset}
\label{tab:enrondegree}
\end{table}

Similarities can be seen with the results of the PageRank algorithm, with the top two nodes being the same, which indicates that these nodes were quite important within Enron.

\subsection{Amazon Product Co-Purchasing}
The Amazon product co-purchasing network\footnote{\url{http://snap.stanford.edu/data/amazon0302.html}} is data collected from the crawling of the Amazon website. It is based on the \emph{Customers Who Bought This Iterm Also Bought} feature of the Amazon website. A node within the network represents a product on sale, and a directed edge between nodes represents that the product has been co-purchased with the product the edge connects to. The data was collected in March 2003 \cite{snap}.

The datasets contains 262,111 nodes and 1,234,877 edges, and is listed with an average clustering coefficient of 0.4240 and 717,719 triangles present.

\subsubsection{PageRank}

\subsubsection{Vertex Degrees}

\subsection{Wikipedia Talk}
The Wikipedia Talk network\footnote{\url{http://snap.stanford.edu/data/wiki-Talk.html}} contains 2,394,385 nodes and 5,021,410 edges. There are 9,203,519 triangles present within the network, and the network has an average clustering coefficient of 0.1958.

\subsubsection{PageRank}

\subsubsection{Vertex Degrees}

\subsection{Patent Citation}
The Patent citation network\footnote{\url{http://snap.stanford.edu/data/cit-Patents.html}} contains 3,774,768 nodes and 16,518,948 edges. There are 7,515,023 triangles present within the network, and the network has an average clustering coefficient of 0.0919.

\subsubsection{PageRank}

\begin{table}%
\begin{tabular}{|l|l|}
\hline
\hline
4237224 & 2.4890060240975007E-5 \\
3813316 & 2.154212981534069E-5 \\
3932805 & 1.2391955911526967E-5 \\
4395486 & 1.1161868281599083E-5 \\
4298685 & 1.0276200370604275E-5 \\
4683195 & 1.0267036400471796E-5 \\
3778614 & 9.558281303049814E-6 \\
4683202 & 9.36024255530041E-6 \\
4358535 & 8.708264966788E-6 \\
3950357 & 8.707317765944523E-6 \\
3789832 & 7.824135128325006E-6 \\
4196265 & 7.4890674305812076E-6 \\
4064521 & 7.458897929020249E-6 \\
3988545 & 7.20869336958276E-6 \\
3763480 & 7.072167214787706E-6 \\
4021726 & 6.992826128596397E-6 \\
3856513 & 6.899726306781281E-6 \\
3747120 & 6.767632714954929E-6 \\
3702886 & 6.693466783697207E-6 \\
2768114 & 6.473626268751898E-6 \\
\hline
\end{tabular}
\caption{Highest PageRank values for Patent Citation network}
\label{tab:patentspr}
\end{table}

\subsubsection{Vertex Degrees}

\subsection{LiveJournal}
The LiveJournal social network\footnote{\url{http://snap.stanford.edu/data/soc-LiveJournal1.html}} contains 4,847,571 nodes and 68,993,773 edges. There are 285730264 triangles present within the network, and the network has an average clustering coefficient of 0.3123.

Unfortunately, this dataset is too large to be analysed on the Hadoop cluster we have access to, so no extra analysis was able to be performed.

\subsection{Twitter}
The large Twitter dataset available indirectly from the SNAP website\footnote{\url{http://snap.stanford.edu/data/twitter7.html}}\footnote{\url{http://an.kaist.ac.kr/traces/WWW2010.html}} consists of 41.7 million nodes and 1.47 billion edges. However, since the LiveJournal dataset was too large to process on the cluster we had available, it was decided not to attempt to process the social network presented.

\subsection{Conclusions}
We have shown that the system we have produced has been able to process social
networks of a much larger size than possible using the previous SNAT tool
developed last year. The actual results from the datasets not directly related
to the intended use of our tool are meaningful to us in anyway, other than
confirming that the tool is working correctly.

It is a shame that the LiveJournal dataset was too large to process, and
similarly so with the large Twitter dataset. These networks represented large
social networks, of which identifying clusters and users of importance would
have been interesting and insightful.

However, due to the approach undertaken in implementing a distributed social
network analysis tool, we know that with a large cluster analysis of these
networks is possible. A larger cluster would also be beneficial with the
analysis of the smaller networks. Executing the PageRank algorithm on the
Patent Citation network took on average twenty-five minutes to complete, which
contrasts with the forty seconds for PageRank to complete on the Enron dataset.

A larger cluster will allow processing of larger datasets, whilst at the same
time, it will also decrease the running time of performing algorithms on
smaller datasets.