\section{Data Sets}
This section should include details about the results of running the algorithms we implemented on each of the datasets we used

\subsection{EDL}
simple enough to do

\subsection{UAF}
simple enough to do

\subsection{Enron}
One dataset which we needed to check the performance of DSNAT with was the Enron email network\footnote{\url{http://snap.stanford.edu/data/email-Enron.html}}. This is a dataset containing nodes representing email addresses, and connections existing between the nodes if at least one email was sent between them. The dataset contains 36,692 nodes and 367,662 edges.

The SNAP website which hosts the Enron dataset provides statistics about each dataset available, including the number of triangles within the dataset (assuming the network is undirected), and the average clustering coefficient.

For the Enron dataset, SNAP lists the number of triangles existing within the network as 727,044 and the average clustering coefficient as 0.4970. After analysis of the network, it became clear the the average clustering coefficient was the network local clustering coefficient.

\subsubsection{3-Clique}
As stated previously, the identification of 3-cliques within a graph is also known as triangle identification. Using the triangle identification algorithm, we were able to mutate the network such that the only nodes and edges remaining within the graph were those nodes constructing at least one triangle two other nodes.

The number of triangles existing within the network is simply the sum of the values stored at all of the nodes within the network, as the node with the highest ID in the triangle contains the count of the triangle.

After summing the number of triangles, we also arrive at the same figure of 727,044 triangles present within the Enron network. Table \ref{tab:enrontriangles} shows the node ID of email addresses present in the Enron network forming the most triangles.

\begin{table}%
\centering
\begin{tabular}{|c|c|}
\hline
Node ID & Number of Triangles \\
\hline
136 & 17744 \\
195 & 15642 \\
76 & 13767 \\
370 & 13671 \\
273 & 13401 \\
1028 & 13064 \\
416 & 11957 \\
292 & 11415 \\
140 & 11265 \\
175 & 10775 \\
734 & 10621 \\
458 & 9217 \\
1139 & 8780 \\
520 & 8662 \\
188 & 8450 \\
444 & 8199 \\
478 & 8068 \\
241 & 7874 \\
639 & 7783 \\
823 & 7590 \\
\hline
\end{tabular}
\caption{Top 20 nodes forming triangles}
\label{tab:enrontriangles}
\end{table}

\subsubsection{Local Clustering Coefficient}
There are 12499 nodes with a local clustering coefficient of 1, which indicates that the direct neighbours of these nodes are also themselves direct neighbours of each other.

There are also 12240 nodes which have a local clustering coefficient of 0, which indicates that nearly one third of the nodes within the network do not form any triangles, and as such are likely to be email addresses external to Enron.

\subsubsection{Network Local Clustering Coefficient}
The network local clustering coefficient is the average of all local clustering coefficients. For the Enron network, DSNAT returns the result of performing the network local clustering coefficient algorithm as: {\tt 0.49698255959950327}.

This value is near identical to the average clustering coefficient provided by with the dataset by SNAP, so shows that our tool, and these implemented algorithms are working correctly.

\subsubsection{PageRank}

\begin{table}%
\centering
\begin{tabular}{|c|c|}
\hline
Node ID & PageRank \\
\hline
5038 & 0.013662506409208037 \\
273 & 0.0032385323056388035 \\
140 & 0.0029987944502075873 \\
458 & 0.0029642208948891463 \\
588 & 0.0029383621879706633 \\
566 & 0.0029094845599689167 \\
1028 & 0.0027870875267837626 \\
1139 & 0.0025456338556696773 \\
370 & 0.0023498594485031215 \\
893 & 0.002199691021020677 \\
195 & 0.002105444989925047 \\
823 & 0.002006310315755743 \\
136 & 0.001883568213027711 \\
286 & 0.0016956990694867655 \\
95 & 0.001694141051220431 \\
5069 & 0.0016638405778172485 \\
5022 & 0.0016023313239255674 \\
76 & 0.0015252624602667306 \\
292 & 0.0015224551091379053 \\
1768 & 0.0015039521949740957 \\
\hline
\end{tabular}
\caption{Top 20 nodes ranked by PageRank values}
\label{tab:enronpagerank}
\end{table}

\subsection{LiveJournal}
needs the cluster working

\subsection{Twitter}
needs the cluster working. This is probably too big for the cluster.