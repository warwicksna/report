\subsection{Clustering Coefficients}
Clustering coefficients provide a way to measure the amount to which vertices within a graph cluster together. Calculating clustering coefficients requires knowing the number of triangles within the graph.

\subsubsection{Global Clustering Coefficient}
The global clustering coefficient is also known as the transitivity of a network. This is because it is similar to the mathematical property of transitivity. A relation ``$\circ$'' is said to be transitive if $a \circ b$ and $b \circ c$ imply together imply $a \circ c$. An example would be the equality relation, as if $a = b$, and $b = c$ it follows that $a = c$ \cite{newman10}. Within a social network, the transitive relation could be described as \textit{``the friend of my friend is also my friend''}.

A global clustering coefficient of 1 implies a social network where all components are cliques. A global clustering coefficient of 0 implies a social network where there are no mutual friends between any two people.

The global clustering coefficient is defined as:

\begin{equation}
C_G = \frac{3 * number\: of\: triangles}{number\: of\: connected\: triples}
\label{eq:globalcc}
\end{equation}

A connected triple is simply three vertices $uvw$ connected with edges ($u, v$) and ($v, w$); edge ($u, w$) does not need to be present \cite{newman10}. The factor 3 is present because each triangle is composed of three connected triples.

\subsubsection{Local Clustering Coefficient}
A clustering coefficient can also be applied to each vertex within a graph. The local clustering coefficient is the probability that vertices connected to a vertex are themselves also connected to each other.

The local clustering coefficient is defined as:

\begin{equation}
C_L = \frac{number\: of\: triangles\: connected\: to\: vertex\: v}{number\: of\: triples\: centered\: on\: vertex\: v}
\label{eq:localcc}
\end{equation}

\subsubsection{Network Local Clustering Coefficient}
The network local clustering coefficient is simply the mean of all local clustering coefficients within a network. It can be used to show the average level of connectedness between vertices in a network.

\begin{equation}
C_N = \frac{1}{n}\sum_{i=1}^{n} C_i
\label{eq:networkcc}
\end{equation}

$C_i$ represents the local clustering coefficient of the vertex $i$. The result of this calculation is a number between 0 and 1, with 1 meaning all vertices are connected to each other \cite{watts98}.
