\section{Emergent Behaviour}

The purpose of this component of the project is to observe and
describe how networks converge and to look at how robust these
networks are in the presence of infiltrators.  More specifically,
we will be looking at networks which use a tag-based approach to
cooperation and the infiltrators will be self-interested agents
which do not follow the social norms of the network.  In our case,
these self-interested agents could be thought of as law-enforcement
individuals.  In general, the presence of self-interested agents
have a negative affect on the cooperation within such a network.
Therefore, we will look at methods to preserve the cooperation rates
in networks where self-interested agents are present.

\subsection{Existing Research}

We are extending the work of Griffiths and Luck (2010) on tag-based
cooperation in a network in the presence of self-interested agents.
This work employs an image-scoring mechanism, network rewiring, and
a learning-based approach to evolution.

The work by Griffiths and Luck extends previous work on tag-based
cooperation, especially Riolo, Cohen, and Axelrod (2001); and Hales
and Edmonds (?).

\subsection{RCA Tag-based Cooperation}

Tag-based cooperation was introduced (Riolo, Cohen and Axelrod,
2001) to observe the nature and method that cooperation in a network
emerges.

The cooperation approach described by RCA takes place in a population
of agents.  Each agent is assigned a single tag and a single tolerance
value.  Both the tag and tolerance are selected randomly from a
uniform distribution between 0 and 1.
The tag is a cultural artefact, a form of a public identifier.
The tolerance is used to determine the willingness of an agent to cooperate.

The approach described by RCA is generational.  Each generation is
divided into two phases: a donation phase and a reproduction phase.
The donation phase occurs first, which allows every agent to perform
a number of donations.  After the donation phase, the reproduction
phase allows agents to reproduce based on their relative success.

During the donation phase, each agent in the population is selected
to make a number of donations.  The number of donations each agent
is selected to make is known as the number of interactions pairings
per generation, P.  When an agent has been selected for donation,
they pick an agent at random to which they have the opportunity to
donate.  In the RCA approach, an agent will donate to the random
agent if the difference between the two agents' tags is less than
the tolerance of the donator.  If an agent chooses to donate, they
will incur a small cost, c, and provide a benefit, b, to the receiving
agent.

\begin{align*}
    |\tau_A - \tau_B| < T_A
\end{align*}

The cooperation of a population of agents is measured by the donation rate in each generation.

Once every agent has been given an opportunity to donate $P$ times,
the reproduction phase takes place.  During the reproduction phase,
each agent reproduces in accordance with their relative success.
The most successful agents produce more offspring than the less
successful agents.  Each agent compares itself to a random agent.
If the agent is less successful than the random agent, then its
offspring derives from the random agent.  If the agent is more
successful than the random agent, then its offspring derives from
iteself.  The offspring takes the tag and tolerance values of the
more successful agent.  There is a small chance that the tag and
tolerance values could be mutated.  If the tag is mutated, the
offspring will receive a new tag taken randomly from the uniform
distribution $\left[0, 1\right]$.  If the tolerance is mutated, a
small amount of gaussian noise with a mean of $0$ is added to the
tolerance value.

Once the donation and reproduction phases for a given generation
have occurred, the process is repeated on the new population.

RCA noted that the donation rate quickly stabilises after around
100 generations.  Occasionally, an agent gets a tolerance value
that is much smaller than the average tolerance as the result of a
mutation.  This mutation allows the agent to become more successful
as they are less likely to donate.  The impact of this reduces the
overall donation rate until the agents converge on the new tag---due
to the reproduction process---at which time the donation rate raises
and stabilises again.

By experimentation, RCA found that with a higher number of interaction
pairings in each generation, the donation rate and average tolerance
increases.  RCA also noted that when the cost of donating became
too high relative to the benefit of receiving a donation, the
donation rate dropped dramatically.  From these results, we will
choose the number of interaction pairings per generation, $P$, to
be at least $3$, the cost of donation to be $0.1$, and the benefit
of receiving a donation to be $1$.

\subsection{Learning-Interpretation of Reproduction}

Hales and Edmonds (2003) extended the notion of tag-based cooperation by RCA
by introducing the notion of a {\emph learning} interpretation instead of a {\emph reproduction} interpretation.

The learning interpretation differs from the {\emph reproduction} interpretation of RCA
in that offspring are not produced in the reproduction phase.
Instead, when an agent is comparing itself to a more successful agent,
they will adopt that agents' tags and tolerances.
This means that the population of agents remains the same,
just the tag and tolerances of an agent change over generations.
Hales and Edmonds showed that the cooperation was not altered by adopting this approach.

Hales and Edmonds also changed the notion of the tag so that
the tag represented the neighbourhood of an agent.
When a new tag is learnt from another agent or through mutation,
the agent in effect rewires its neighbourhood to remove connections to those agents with its old tag,
and forging connections to agents with the same new tag.
In Hales and Edmonds' approach, donations are only permitted between agents in the same neighbourhood.

\subsection{Image Scoring}

Griffiths (2008) extended the cooperation approach of RCA,
using the learning interpretation of HE, by connecting agents
in a graph.
All agents are connected in a graph with a random network topology.
Agents are only able to donate to agents to which they are connected.

This model was used to observe the effect of self-interested agents (cheaters)
on the donation rate.
A cheater is an agent that never chooses to donate, despite the tags and tolerance values.
As such, a cheater is likely to be very successful as they
receive the benefits of donation without incurring the costs.
It was shown that the presence of just a small amount of cheaters in the network
caused a large reduction in the overall rates of cooperation.

To address this problem, image scoring was introduced.
In image scoring, each agent can observe the interation pairings of its neighbours.
Therefore,
and agent can keep track of the times when a neighbour chooses to donate and refuses to donate.
These observations for each neighbour are stored in a queue data structure.
When an agent is selected to make a donation, they will consider the observations of their local network
when donating in addition to their tag and tolerance values.

This approach to ranking a neighbourhood is chosen
due to the lack of direct reciprocity.
Direct reciprocity is when, by donating to another agent, it is likely that the other agent will in future donate to you.
It is therefore important to consider the neighbourhood to which an agent belongs when they make a donation.

\begin{align*}
    T_A' = T_A + \left(T_A \times \frac{\sum_{i = 1}^{n}\delta_i}{n}\right)
\end{align*}

Until each agent has recorded enough observations,
the basic tag and tolerance approach described by RCA is used.

It was observed that using this approach,
a significant increase in the donation rate, particularly
with small percentage of cheaters, could be achieved.

\subsection{Network Rewiring}

In order to further increase cooperation
in the presence of cheaters, network rewiring was introduced (Griffiths and Luck, 2010).
This allows an agent to rewire their connections to other agents,
in effect changing the neighbourhood of an agent.
The purpose of this is to allow an agent to remove connections
to agents which are unlikely to donate and forge connections to agents
which are more likely to donate.

Each agent which learnt during the learning phase,
is given an opportunity to rewire their own connections.
At this point, each of these agents is allowed to drop any number of its
connections to other agents and then add connections to new neighbours.
The aim of this is to allow agents to remove un-cooperative agents from
their neighbourhood in order to increase cooperation in their network.

There are two parameters that are used to control the rewiring phase:
the rewire proportion and the rewire strategy.
The rewire proportion is a percentage which is used to determine the number of
neighbours removed or added during a rewiring.
Whent the rewire proportion is zero, then no rewiring takes place.
When the rewire proportion is one, then all of an agents connections will be dropped and replaced with new connections.
There is a single rewire proportion value for all agents, which remains constant throughout the whole simulation.

The rewiring strategies determine which of an agents' connections are dropped and which new connections are to be acquired.
In their paper, Griffiths and Luck introduced four different rewiring strategies,
these are: Random, Random Replace Worst, Individual Replace Worst and Group Replace Worst.
For the purposes of the following descriptions of these strategies,
it is assumed that each neighbour of an agent can be ranked in order of their observed past donation behaviour.
This assumption is used to allow a rewiring strategy to identify the most and least cooperative agents in an agent's network.

\begin{description}
\item[Random Rewiring Strategy]
When the random rewiring strategy is used,
each agent will drop a proportion of their connections at random.
Once an agent has dropped these connections, the agent will then add new
connections to new agents at random to replace the connections dropped.

\item[Random Replace Worst Rewiring Strategy]
In this strategy, each agent will rank their neighbouring agents in terms
of their perceived willingness to donate, based on past observations.
Based on these rankings, the rewiring agent will drop their neighbours
which have the worst donation rate track record. Once these connections have
been dropped, random connections to new neighbours will be forged---as in
the random rewiring strategy.

\item[Individual Replace Worst Rewiring Strategy]
The individual replace worst strategy will---like the random replace
worst rewire strategy---remove connections to their neighbours which have
the worst record of donations. After these connections have been dropped,
the agent will look at their most willing neighbour and replace the lost
connections with their neighbour's best ranked connections. In the case where
adding a new connection would duplicate an existing connection, or connect an
agent to itself, a random connection is added instead. The effect of this is
that the new connections are the same as the best connections of their best
neighbour.

\item[Group Replace Worst Rewiring Strategy]
The group replace worst rewiring strategy is very similar to the individual
replace worst rewiring strategy. Like the individual replace worst rewiring
strategy, the neighbouring agents with the worst donation track record are
dropped. Once these connections have been dropped, the agent will connect
to the best ranked neighbour of each of its best ranked neighbours. As in the
individual replace worst rewiring strategy, any possible duplicate connections
will instead by replace by a random new connection.

\end{description}

\subsection{Effects of Rewiring Strategy on Cooperation}

To observe the effect of these different rewiring strategies as well as the
rewire proportion on the donation rate, Griffiths and Luck performed a number
of experiments.

It was found that the random rewiring strategy performed better than the RCA
approach, but worse than the RCA approach augmented with context assessment
(Griffiths, 2008). It was found that all the other rewiring strategies resulted
in a better donation rate than either of the RCA approaches. It was found that
rewiring was poorest when the rewiring proportion was close to zero or one.
It was found that the random replace worst rewwiring strategy performed poorly
when the rewiring proportion was greater than around $0.6$. It was found that
the individual and group replace worst rewiring strategies performed similarly
and had the best results when the rewire proportion was between $0.4$ and $0.8$.

From the results, it was show that by implementing these simple rewiring
strategies, it is possible to get a significant increase in overall cooperation---
in the paper, there was an increase of around 20\% over the image-scoring approach
in populations with 10\%, 20\%, and 30\% cheaters.

From these results, it appears to be sensible to choose a rewire proportion in
the range $0.4$ to $0.8$. In the paper, it was suggested to use a value of $0.6$
for the rewire proportion.
