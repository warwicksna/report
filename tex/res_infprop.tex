\subsection{Influence Propogation}

 Influence propogation algorithms attempt to model the spread of ideas, beliefs, viruses, or other transmissable entities, through a population. A social network is modeled as a graph, with nodes representing individuals, and vertices representing relationships between individuals. In most influence propogation models, nodes can be in one of two states: "converted" or "unconverted". (!change or cite following sentence) Such a model can represent the adoption of new ideas (Rogers 2003), the spread of infectious disease (Anderson and May 1991), word-of-mouth recommendations (Goldenberg, Libai and Muller 2001), viral marketing campaigns (Kempe, Kleinberg and Tardos 2003) and information cascades in online social networks (Lerman and Ghosh 2010).

 The influence propagation model must also describe how influence spreads over time; that is, the conditions that cause a node to change states from "unconverted" to "converted". There are two main models of transmission used; the independent cascade model, and the linear threshold model.

 In the independent cascade model, whenever a node becomes converted, it has one chance to convert each of its neighbours. Every edge (A -> B) is assigned a weight, which represents the probability that node A can convert node B, or vice versa.

 In the linear threshold model, every node has a threshold value t. When the number of converted neighbours is greater than t, the node becomes converted. Again, edges can be weighted, in which case the condition for conversion is <maths>.

 There are limitations to the two-state model. For example, it cannot represent a scenario where two competing ideas spread through a network. <paper> studies a model for such situations where nodes can be in one of 3 states; "unconverted", "converted to idea A", or "converted to idea B". In this case, A and B represent opposing beliefs.
