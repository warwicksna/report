\chapter{Research}
This chapter is composed of three sections. The first concerns itself with existing research within the area of the analysis of social networks. The second . The final section is concerned with distributed computing, and how it can be used for analysing graphs of social networks.

% Since we have a few distinct sections on this area, I think it's a good idea to split this into multiple smaller files as well

\section{Social Network Analysis}
Analysis of social networks is becoming ever-increasing in importance, with the huge growth social network sites such as Facebook\footnote{\url{https://www.facebook.com/}} and Twitter\footnote{\url{http://twitter.com/}} have experienced in recent years. This research undertaken within this section relates to various metrics which can be applied to a social network to identify important people within the network, and also how the structure of the network can be identified.

\subsection{Community Detection}
An important aspect of social network analysis is the detection of communities within a graph.

\subsubsection{Clustering}
Clustering is one approach to identifying communities within a social network. A cluster within a social network represents a group of people who do not share many interactions with other people outside of their cluster.

\subsubsection{\emph{k}-Cliques}
The term $clique$ is introduced by \citeauthor{luce49} in \cite{luce49} to describe a subgraph which consists of at least three vertices, each of which are fully connected with each other. From a social network perspective, this translates to saying that for each person in the clique, all of their friends  within the clique are friends of each other as well.

A \emph{k}-clique

\subsubsection{\emph{k}-Trusses}

\subsubsection{Clustering Coefficients}
Clustering coefficients provide a way to observe the 

\subsection{Clustering Coefficients}
Clustering coefficients provide a way to measure the amount to which vertices within a graph cluster together. Calculating clustering coefficients requires knowing the number of triangles within the graph.

\subsubsection{Global Clustering Coefficient}
The global clustering coefficient is also known as the transitivity of a network. This is because it is similar to the mathematical property of transitivity. A relation ``$\circ$'' is said to be transitive if $a \circ b$ and $b \circ c$ imply together imply $a \circ c$. An example would be the equality relation, as if $a = b$, and $b = c$ it follows that $a = c$ \cite{newman10}. Within a social network, the transitive relation could be described as \textit{``the friend of my friend is also my friend''}.

A global clustering coefficient of 1 implies a social network where all components are cliques. A global clustering coefficient of 0 implies a social network where there are no mutual friends between any two people.

The global clustering coefficient is defined as:

\begin{equation}
C_G = \frac{3 * number\: of\: triangles}{number\: of\: connected\: triples}
\label{eq:globalcc}
\end{equation}

A connected triple is simply three vertices $uvw$ connected with edges ($u, v$) and ($v, w$); edge ($u, w$) does not need to be present \cite{newman10}. The factor 3 is present because each triangle is composed of three connected triples.

\subsubsection{Local Clustering Coefficient}
A clustering coefficient can also be applied to each vertex within a graph. The local clustering coefficient is the probability that vertices connected to a vertex are themselves also connected to each other.

The local clustering coefficient is defined as:

\begin{equation}
C_L = \frac{number\: of\: triangles\: connected\: to\: vertex\: v}{number\: of\: triples\: centered\: on\: vertex\: v}
\label{eq:localcc}
\end{equation}

Within social network analysis, the local clustering coefficient

\subsubsection{Network Local Clustering Coefficient}
The network local clustering coefficient is simply the mean of all local clustering coefficients within a network. It can be used to show the average level of connectedness between verticies in a network.

\begin{equation}
C_N = \frac{1}{n}\sum_{i=1}^{n} C_i
\label{eq:networkcc}
\end{equation}

$C_i$ represents the local clustering coefficient of the vertex $n_i$. The result of this calculation is a number between 0 and 1, with 1 meaning all vertices are connected to each other \cite{watts98}.

\subsection{Centrality}
Following from identifying communities within a social network, it is also apparent that establishing the most \emph{central} or \emph{important} person is within a community. This person is most likely going to have a large influence over the members of the community 

\subsubsection{}
\subsection{Influence Propogation}

 Influence propogation algorithms attempt to model the spread of ideas, beliefs, viruses, or other transmissable entities, through a population. A social network is modeled as a graph, with nodes representing individuals, and vertices representing relationships between individuals. In most influence propogation models, nodes can be in one of two states: "converted" or "unconverted". (!change or cite following sentence) Such a model can represent the adoption of new ideas (Rogers 2003), the spread of infectious disease (Anderson and May 1991), word-of-mouth recommendations (Goldenberg, Libai and Muller 2001), viral marketing campaigns (Kempe, Kleinberg and Tardos 2003) and information cascades in online social networks (Lerman and Ghosh 2010).

 The influence propagation model must also describe how influence spreads over time; that is, the conditions that cause a node to change states from "unconverted" to "converted". There are two main models of transmission used; the independent cascade model, and the linear threshold model.

...

\subsubsection{Independent Cascade}
 In the independent cascade model, whenever a node becomes converted, it has one chance to convert each of its neighbours. Every edge (A -> B) is assigned a weight, which represents the probability that node A can convert node B, or vice versa.

...

\subsubsection{Linear Threshold}
 In the linear threshold model, every node has a threshold value t. When the number of converted neighbours is greater than t, the node becomes converted. Again, edges can be weighted, in which case the condition for conversion is <maths>.

 There are limitations to the two-state model. For example, it cannot represent a scenario where two competing ideas spread through a network. <paper> studies a model for such situations where nodes can be in one of 3 states; "unconverted", "converted to idea A", or "converted to idea B". In this case, A and B represent opposing beliefs.

...

\section{Influence Manipulation}

\subsection{Recommender Systems}

Although relatively little publicly available research exists on the topic of manipulating social networks, there is a wealth of information on attack and defence of recommender systems. Recommender systems are a tool used to tackle the problem of information overload; where more content exists than a user has time to evaluate. As the name suggests, they do so by performing an initial evaluation on behalf of the user, whittling the quantity of presented content down to a level the user can cope with. Amazon and Twitter them to recommend products and followers respectively. They are fundamentally similar to social networks in the underlying graph structure, the attempt to model a real-world phenomena, and the clear incentives various parties have to manipulate them.

\subsubsection{Attacker Goals}

Collaborative Recommendation: A Robustness Analysis defines two simple intents an attacker may have: `nuking' an item (decreasing its rating / the frequency with which it is recommended) and `pushing' an item (increasing its rating). `Vandalising'; making an entire system function poorly, is discussed in another paper. Every one of these intents corresponds to something one may wish to achieve in a real influence network; altering an entity's influence or disrupting an entire network.

\subsubsection{k-Nearest-Neighbour}

The actual attacks used to achieve a given goal vary greatly depending on the implementation of the target system, but the base components remain the same. Early recommender systems such as Tapestry used a variation of the k-Nearest-Neighbour algorithm (see page x) to identify like-minded people to new users, and uses this as a basis for their recommendations.  

\begin{equation*}
pu,i = ru + \Sigma v \in Uu,i [wu,v (rv,i − rv)]
\end{equation*}

\begin{equation*}
\Sigma v \in Uu,i \left\|wu,v \right\|
\end{equation*}
This is by design not dissimilar to the reputation and influence systems
underlying the networks that we want to tackle. The item-item algorithm's
fundamental difference is that it directly calculates the distance between items
rather than between users. ``People who liked X also liked Y''

\subsubsection{Attacks}

The paper `Shilling Recommender Systems for Fun and Profit' considers attacks in terms of their intent, targets, required knowledge and cost. There is a cost per shill whether they are an undercover agent, the product of a persona management system or simply a solved captcha (the latter being the only part of account-creation on a website that can't be automated, and costing \$1.39/1000 from deathbycaptcha.com ). The simplest push/nuke attacks revolve around gaining as many neighbours as possible, then rating the target item. The simple but ineffective approach to the linking stage taken by the `RandomBot' is to assign a random rating to all the items in the database. The `AverageBot' attack instead assigns to each item the average rating it's received so far. This has a lower cost at the expense of a greater knowledge requirement. Even on a network like Twitter where most data is publicly available, there is a cost associated with acquiring and processing it. Toward Trustworthy Recommender Systems: An Analysis of Attack Models and Algorithm Robustness introduces two additional, more implementation specific attacks. The bandwagon attack achieves a high level of association for a low cost by rating a small number of frequently and consistently rated items (e.g. bestselling novels).  The segment attack is simply a refinement that only attempts to promote items to those already disposed to buying them.

\begin{itemize}
	\item +love/hate
	\item reverse bandwagon
\end{itemize}

These attacks provide us with a number of starting techniques, as well as an insight into something

\subsubsection{Defences}

The defences are also relevant. It's entirely plausible that as these attacks
are turned towards infiltrating and manipulating social networks, parties with a
vested interest in protecting the network's integrity will look at the
compatriot countermeasures; attack detection and attack mitigation. Shill
detection algorithms are already in use on Twitter to detect and remove the
original shills; spambots. It may be well beyond the scope of this project, but
that doesn't diminish the value of a design informed by an understanding of
future problems.

\subsubsection{Mitigation}

One mitigation that is at least conceptually simple  is to use a hybrid model
where a significant portion of the model isn't taken from user data. For
example, semantically enhanced collaborative modelling uses text-mining
techniques to create a similarity measure entirely independent of user input.
From the perspective of a persona management system, all target systems use a
hybrid of malleable online input and untouchable real-world interactions.
Physically deployed undercover operatives have almost the inverse restriction;
excessive manipulation of online systems may raise suspicion, something
distinctly more perilous for those physically present.

\subsubsection{Detection}
Attacks may be detected using simple metrics such as analysing user's rating
deviation and profile length; the randomBot attack has a clear footprint in this
regard. Perhaps in recognition of the inherent inaccuracy of such measures, once
a profile is classified as ``suspicious'' its ratings are simply ignored, rather
than it being booted off the system as might happen to a hacker on an attack
aware application.

\begin{itemize}
	\item rating deviation from main agreement
	\item they act in groups
\end{itemize}


\url{http://blog.mozilla.org/webappsec/2011/02/02/attack-aware-applications/}

Original: \url{http://citeseerx.ist.psu.edu/viewdoc/download?doi=10.1.1.13.2238\&rep=rep1\&type=pdf}

Fun\&profit: \url{http://citeseerx.ist.psu.edu/viewdoc/download?doi=10.1.1.65.579\&rep=rep1\&type=pdf}

Latest: \url{http://maya.cs.depaul.edu/~mobasher/papers/mbbw-acmtoit-07.pdf}
\section{Emergent Behaviour}

The purpose of this component of the project is to examine and
describe how networks of agents converge and cooperate, and to look at measuring the robustness of this cooperation of such
networks in the presence of infiltrators.  More specifically,
we will be looking at networks which use a tag-based approach to
cooperation based on the work done by Riolo, Cohen, Axelrod (RCA) \cite{rca}.
The infiltrators will be self-interested agents, agents
which do not follow the social norms of the network.  In our case,
these self-interested agents could be thought of as law-enforcement
individuals, who may be interested in joining such a network but not cooperate.
In general the presence of self-interested agents
have a negative affect on the cooperation within such a network \cite{aamas2008}.
We will therefore look at methods to preserve the high rates of cooperation
in these networks when self-interested agents are present.

\subsection{Existing Research}

The work will be based on a number of existing pieces of research on tag-based cooperation within a network of agents in the presence of self-interested agents.

\subsection{RCA Tag-based Cooperation}

Tag-based cooperation was introduced to observe the nature and method that cooperation in a network
emerges \cite{rca}.
The purpose of this research was to observe how cooperation can be achieved in a network where there is no direct reciprocity.

The cooperation approach described by RCA takes place in a population
of agents.  Each agent is assigned a single tag and a single tolerance
value.  Both the tag, $\tau$, and tolerance, $T$, are selected randomly from a
uniform distribution between $0$ and $1$.
The tag is a cultural artefact---a form of a public identifier.
The tolerance is used to determine the willingness of an agent to cooperate.

The approach described by RCA is generational.  Each generation is
divided into two phases: a donation phase and a reproduction phase.
The donation phase occurs first and allows every agent to perform
a number of donations.  After the donation phase the reproduction
phase allows agents to reproduce based on the relative success of each agent---more successful agents will produce a larger amount of offspring.

During the donation phase, each agent in the population is selected
to make a number of donations.  The number of donations each agent
is selected to make is known as the number of interactions pairings
per generation, $P$.  When an agent has been selected for donation
an agent is picked at random to which they have the opportunity to
donate.  In the RCA approach an agent will donate to the random
agent if the difference between the two agents' tags is less than
the tolerance of the donator, using the formula presented in \ref{fig:rca-donation-eqn}.  If an agent chooses to donate, they
will incur a small cost, $c$, and provide a benefit, $b$, to the receiving
agent.

\begin{figure}[htbp]
	\begin{equation*}
    	|\tau_A - \tau_B| < T_A
	\end{equation*}
	\caption{Equation for determining if an agent will donate}
	\label{fig:rca-donation-eqn}
\end{figure}

The cooperation of a population of agents is measured by the donation rate in each generation.

Once every agent has been given an opportunity to donate $P$ times,
the reproduction phase takes place.  During the reproduction phase,
each agent reproduces in accordance with their relative success.
The most successful agents produce more offspring than the less
successful agents.  Each agent compares itself to a randomly selected agent.
If the agent is less successful than the randomly selected agent then its
offspring derives from the random agent.  If the agent is more
successful than the random agent, then its offspring derives from
itself.  The offspring takes the tag and tolerance values of the
more successful agent.  There is a small chance that the tag and
tolerance values could be mutated.  If the tag is mutated, the
offspring will receive a new tag taken randomly from the uniform
distribution $\left[0, 1\right]$.  If the tolerance is mutated, a
small amount of gaussian noise with a mean of $0$ is added to the
tolerance value.

Once the donation and reproduction phases for a given generation
have occurred, the process is repeated on the new population.

In experiments, RCA observed that the donation rate quickly stabilises after around
$100$ generations.  Occasionally, an agent gets a tolerance value
that is much smaller than the average tolerance as the result of a
mutation.  This mutation allows the agent to become more successful
as they are less likely to donate.  The impact of this reduces the
overall donation rate until the agents converge on the new tag---due
to the reproduction process---at which time the donation rate raises
and stabilises again.

By experimentation, RCA found that with a higher number of interaction
pairings in each generation, the donation rate and average tolerance
increases.  RCA also noted that when the cost of donating became
too high relative to the benefit of receiving a donation, the
donation rate dropped dramatically.  From these results, we will
choose the number of interaction pairings per generation, $P$, to
be at least $3$, the cost of donation to be $0.1$, and the benefit
of receiving a donation to be $1$.

While RCA chose to use a generational approach where every agent produces offspring,
an alternative approach based around a learning interpretation was mentioned but not implemented.

\subsection{Learning-Interpretation of Reproduction}

Hales and Edmonds (HE) applied the learning interpretation presented by RCA \cite{he}.

The learning interpretation differs from the {\emph reproduction} based interpretation of RCA
in that no offspring are not produced in the reproduction phase.
Instead, if during the reproduction phase an agent compares itself to a randomly selected agent which is more successful than itself, the agent will adopt the randomly selected agents' tag and tolerance values.
This means that the population of agents remains the same but the tag and tolerances of the agents change in each generation.
HE demonstrated that the rate of cooperation was not altered by adopting this approach.

Another change HE made was to make a tag represent the neighbourhood of the agent.
When a new tag is learnt from another agent or through mutation,
this new tag effectively causes the agent to rewire itself in a network so as to remove connections to those agents with its old tag,
and forge connections to agents that share the same new tag.
In the HE approach, donations are only permitted between agents in the same neighbourhood.
Using this interpretation, a tag could be thought of as a gang sign, and agents would only have a chance to cooperate with those agents in the same gang.

\subsection{Image Scoring}

In 2008, the notion of tag-based cooperation was altered to observe how cooperation changes in the presence of self-interested agents \cite{aamas2008}.
The basic approach of RCA was combined with the learning interpretation of HE.
In addition, all the agents were connected in a graph with a random network topology.
Agents were only able to donate to the agents to which they are connected.

This model was used to observe the effect of self-interested agents (cheaters)
on the donation rate.
In this case, a cheater is an agent that always refuses to donate to other agents, despite the tag and tolerance values.
As such a cheater is likely to be very successful, as they receive the benefits of donation without incurring the costs.
It was shown that the presence of just a small amount of cheaters in the network
caused a large reduction in the overall rates of cooperation.

In order to address this problem, an image scoring mechanism was introduced.
Every agent observes the interaction pairings of its neighbours.
Each agent keeps track of the times when a neighbouring agent chooses or refuses to make a donation.
The observations for each neighbour are stored in a queue data structure.
When an agent is selected to make a donation, they will consider the observations of their local network
when donating in addition to their tag and tolerance values.

This approach to ranking a neighbourhood is chosen due to the lack of direct reciprocity.
Direct reciprocity is when, by donating to another agent, it is likely that the other agent will in future donate to you.
In these such networks however, direct reciprocity does not occur,
it was therefore considered important to take into account the neighbourhood to which an agent belongs when they make a donation.
The choice of an agent to donate is determined by the equation \ref{fig:aamas-2008-donation-eqn}.

\begin{figure}[htbp]
\begin{align*}
    T_A' = T_A + \left(T_A \times \frac{\sum_{i = 1}^{n}\delta_i}{n}\right)
\end{align*}
\caption{Equation to determine if an agent is ready to donate, augmented with neighbourhood assessment}
\label{fig:aamas-2008-donation-eqn}
\end{figure}

Until each agent has recorded enough observations,
the basic tag and tolerance approach described by RCA is used.

It was observed that using this approach,
a significant increase in the donation rate, particularly
with small percentage of cheaters, could be achieved.

\subsection{Network Rewiring}

In order to further increase cooperation in the presence of cheaters,
the ability for agents to rewire their connections was introduced \cite{aamas2010}.
This allows an agent to rewire their connections to other agents, in effect changing the neighbourhood of an agent.
The reasoning behind this was to allow an agent to drop connections to the agents in their neighbourhood which are unlikely to donate, and replace these with connections to new agents which are more likely to donate.

Each agent which learnt during the learning phase,
is given an opportunity to rewire their own connections.
At this point, each of these agents is allowed to drop any number of its
connections to other agents and then add connections to new neighbours.
The aim of this is to allow agents to remove un-cooperative agents from
their neighbourhood in order to increase cooperation in their network.

There are two parameters that are used to control the rewiring phase:
the rewire proportion and the rewire strategy.
The rewire proportion is a percentage which is used to determine the number of
neighbours removed or added during a rewiring.
When the rewire proportion is zero, then no rewiring takes place.
When the rewire proportion is one, then all of an agents connections will be dropped and replaced with new connections.
There is a single rewire proportion value for all agents, which remains constant throughout the whole simulation.

The rewiring strategies determine which of an agents' connections are dropped and which new connections are to be acquired.
In their paper, Griffiths and Luck introduced four different rewiring strategies,
these are: Random, Random Replace Worst, Individual Replace Worst and Group Replace Worst.
For the purposes of the following descriptions of these strategies,
it is assumed that each neighbour of an agent can be ranked in order of their observed past donation behaviour.
This assumption is used to allow a rewiring strategy to identify the most and least cooperative agents in an agent's network.

\begin{description}
\item[Random Rewiring Strategy]
When the random rewiring strategy is used,
each agent will drop a proportion of their connections at random.
Once an agent has dropped these connections, the agent will then add new
connections to new agents at random to replace the connections dropped.

\item[Random Replace Worst Rewiring Strategy]
In this strategy, each agent will rank their neighbouring agents in terms
of their perceived willingness to donate, based on past observations.
Based on these rankings, the rewiring agent will drop their neighbours
which have the worst donation rate track record. Once these connections have
been dropped, random connections to new neighbours will be forged---as in
the random rewiring strategy.

\item[Individual Replace Worst Rewiring Strategy]
The individual replace worst strategy will---like the random replace
worst rewire strategy---remove connections to their neighbours which have
the worst record of donations. After these connections have been dropped,
the agent will look at their most willing neighbour and replace the lost
connections with their neighbour's best ranked connections. In the case where
adding a new connection would duplicate an existing connection, or connect an
agent to itself, a random connection is added instead. The effect of this is
that the new connections are the same as the best connections of their best
neighbour.

\item[Group Replace Worst Rewiring Strategy]
The group replace worst rewiring strategy is very similar to the individual
replace worst rewiring strategy. Like the individual replace worst rewiring
strategy, the neighbouring agents with the worst donation track record are
dropped. Once these connections have been dropped, the agent will connect
to the best ranked neighbour of each of its best ranked neighbours. As in the
individual replace worst rewiring strategy, any possible duplicate connections
will instead by replace by a random new connection.

\end{description}

\subsection{Effects of Rewiring Strategy on Cooperation}

To observe the effect of these different rewiring strategies as well as the
rewire proportion on the donation rate, Griffiths and Luck performed a number
of experiments.

It was found that the random rewiring strategy performed better than the RCA
approach, but worse than the RCA approach augmented with context assessment
(Griffiths, 2008). It was found that all the other rewiring strategies resulted
in a better donation rate than either of the RCA approaches. It was found that
rewiring was poorest when the rewiring proportion was close to zero or one.
It was found that the random replace worst rewiring strategy performed poorly
when the rewiring proportion was greater than around $0.6$. It was found that
the individual and group replace worst rewiring strategies performed similarly
and had the best results when the rewire proportion was between $0.4$ and $0.8$.

From the results, it was show that by implementing these simple rewiring
strategies, it is possible to get a significant increase in overall cooperation---
in the paper, there was an increase of around 20\% over the image-scoring approach
in populations with 10\%, 20\%, and 30\% cheaters.

From these results, it appears to be sensible to choose a rewire proportion in
the range $0.4$ to $0.8$. In the paper, it was suggested to use a value of $0.6$
for the rewire proportion.


\section{Distributed Computing}
With the huge sizes social networks can reach, there needs to be efficient approaches to decrease the running time of algorithms for analysis of the networks. One approach is to use distributed computing to parallelise these algorithms, so that computation across the network occurs at the same time, where possible, and as such should reduce the running time of these algorithms.

One area of distributed computing which has been growing in recently is the use of the MapReduce framework developed by Google, and the free implementation by Apache of MapReduce, Hadoop. We will be looking to use Hadoop as an approach to parallelise algorithms for social network analysis.

This should achieve a reduced running time for algorithms, and also allow analysis of far larger networks than previously possible due to the scalability of the Hadoop framework.

\subsection{MapReduce}
MapReduce is a programming framework designed to simply processing data on large clusters \cite{mapreduce}. MapReduce works by the user supplying two functions, Map and Reduce, which operate in parallel on the data set provided. The Map function processes data from key/value pairs into an intermediate set of key/value pairs, before the Reduce function processes this intermediate set into final key/value pairs.

\lstset{language=C++,caption={Calculating the frequency of words in files using MapReduce \cite{mapreduce}},label=lst:mapreduceexample,tabsize=2,breaklines=true,breakatwhitespace=true,frame=single}
\begin{lstlisting}[float]
map(String key, String value):
	// key: document name
	// value: document contents
	for each word w in value:
		EmitIntermediate(w, ``1'');
		
reduce(String key, Iterator values):
	// key: a word
	// values: a list of counts
	int result = 0;
	for each v in values:
		result += ParseInt(v);
	Emit(AsString(result));	
\end{lstlisting}

Listing \ref{lst:mapreduceexample} is an example MapReduce program which counts the frequency of words within a selection of documents stored on a system. The Map function reads each word, \emph{w}, and emits the word as key/value pair \verb/(w, ``1'')/ to signify that there is an occurrence of \emph{w} at that position. The Reduce function sums together the value each of these emitted key/value pairs, where the key is the same, and emits the sum for each word.

Whilst the example given in Listing \ref{lst:mapreduceexample} uses Strings for the input and output for both of the Map and Reduce functions, it is not necessarily the case that all Map and Reduce functions operate in this way. \cite{mapreduce} explains that the types used by both are linked, as shown by:

\begin{verbatim}
map     (k1, v1)        -> list(k2, v2)
reduce  (k2, list(v2))  -> list(v2)
\end{verbatim}

This states the input keys and values are from a different domain to the output keys and values, which also means that the types used can differ.

\subsubsection{Google File System}
The Google File System, GFS, provides the distributed file system which MapReduce operates with \cite{mapreduce}, but was developed outside of MapReduce to address issues found with previous distributed file systems \cite{gfs}.

The GFS was designed to meet three major points identified with existing distributed file systems \cite{gfs}:
\begin{enumerate}
	\item Component failures are the norm, rather than the exception
	\item Files are huge by traditional standards
	\item Files are mutated by appending new data
\end{enumerate}

As hardware failures are common, the design of the GFS incorporates this, and the system is monitoring itself continually to detect, tolerate, and recover promptly from component failures on a routine basis \cite{gfs}. In addition to this, the system used by Google makes use of inexpensive hardware due to the frequent failures experienced, and as such is a more cost-effective solution than using more expensive tailored hardware.

A GFS cluster is split into a single $master$ and multiple $chunkservers$ and is accessed by multiple $clients$ \cite{gfs}. Files stored in the GFS are split into chunks, which are stored across the cluster on the hard disks located on each chunkserver. By default, each chunk is replicated in the file system three times for reliability of access to data within file system.

Files are stored into the GFS in chunks of size 64MB. This size was chosen to reduce the need to interact with master to find the location of chunks to read data from, and write data to. The larger chunk size also reduces the quantity of metadata stored on the master, which increases the performance of the master as the metadata can be stored in memory, reducing lookup times \cite{gfs}.

The master node maintains the metadata for the file system. This includes the locations of chunks across the file system, and which chunks compose the files stored. The master node communicates with each chunkserver frequently, and if it does not receive a response, the chunkserver is deemed to have failed and any chunks which are then under replicated in the file system are re-replicated to ensure that the minimum number of replications for each chunk are observed.

\subsection{Hadoop}
Hadoop\footnote{\url{http://hadoop.apache.org/}} is framework for performing
distributed computing. It is a free implementation of the MapReduce framework
developed at Google, and is also a top-level project hosted by Apache.

Hadoop has diversified itself from its conception, and is now composed of three
subprojects, Hadoop Common, Haddop Distributed File System, and Hadoop
MapReduce. Hadoop Common providse common utilities which support the other
Hadoop subprojects. The Hadoop Distributed File System is described in more
detail in Section \ref{sec:hdfs}

Hadoop MapReduce is the subproject by which Hadoop itself more known for. It
provides functionality similar to the MapReduce framework developed by Google,
where there exists a $Map$ and a $Reduce$ function which process data across a
cluster.

\subsubsection{Hadoop Distributed File System}
\label{sec:hdfs}
Hadoop also provides a the Hadoop Distributed File System, HDFS. The HDFS is a free implementation of the Google File System, and is designed to be used with Hadoop itself, though can also be used a distributed file system by itself \cite{hdfs}.

The HDFS operates in a similar approach to the operation of Hadoop and the Google File System. There exists a master node, called the NameNode, and many slave nodes, called DataNodes. The NameNode co-ordinates the access of files stored in the HDFS, and also manages the file system namespace. There is usually a DataNode present on each physical node within the cluster. It is the DataNode which control the storage of files on the storage system present on the node, and also controls the reading and writing of files to the HDFS from a user.

The HDFS ensures data integrity through replication of data across different nodes within the cluster. A replication factor is set for the cluster, generally at least 3, which causes all data within the HDFS to be replicated at least that many times. Data within the HDFS is split into blocks, with a large file being represented by many smaller blocks, and it is these blocks which are replicated across the HDFS.

In case of a problem with the HDFS, such as a partial network failure, or hard disk failure, each DataNode is required to periodically message the NameNode which contains a report on all data blocks stored by that DataNode. If there is a failure of some kind, then the NameNode either receives an incomplete message or no message at all. This informs the NameNode that there is a problem with the HDFS and takes appropriate action, including re-replicating the lost data from other DataNodes to new DataNodes.

The HDFS also provides another service called the SecondaryNameNode. The SecondaryNameNode is not a direct failover service to the NameNode. Instead, the SecondaryNameNode takes periodic checkpoints of the state of the NameNode, so that in case of the NameNode failing, a recent copy of the state of the HDFS can be loaded when the NameNode is restarted, which should result in minimal problems with resuming the HDFS.
\subsection{Giraph}
Giraph is a free implementation of Google's Pregel framework.
